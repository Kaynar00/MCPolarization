\documentclass[12pt]{amsart}
\usepackage{geometry}                % See geometry.pdf to learn the layout options. There are lots.
 \geometry{
 a4paper,
 total={170mm,257mm},
 left=20mm,
 top=20mm,
 }
\usepackage{graphicx}
\usepackage{amssymb}
\usepackage{amsmath, amsthm}
\usepackage{epstopdf}
\usepackage{array}
\usepackage{url}
\usepackage{color}

\DeclareGraphicsRule{.tif}{png}{.png}{`convert #1 `dirname #1`/`basename #1 .tif`.png}

\newcolumntype{L}[1]{>{\raggedright\let\newline\\\arraybackslash\hspace{0pt}}m{#1}}
\newcolumntype{C}[1]{>{\centering\let\newline\\\arraybackslash\hspace{0pt}}m{#1}}
\newcolumntype{R}[1]{>{\raggedleft\let\newline\\\arraybackslash\hspace{0pt}}m{#1}}
\newcommand{\todo}[1]{{\color{red}$\blacksquare$~\textsf{[TODO: #1]}}}

\begin{document}

\noindent
\begin{tabular}{L{9.7cm} R{6.7cm}}
Astrophysical Radiation Processes: AS7005& Lab: 26.09.2022 \\
Evan O'Connor \& Yutong He & Due: 31.10.2022; on Athena\\
\end{tabular}\\

\vspace*{0.5cm}

\centerline{\Large \underline{Introduction to AS7005 Laboratory Exercise} }
\vspace*{0.5cm}

The purpose of this laboratory exercise is to solidify your
understanding of of the fundamental radiation theory concept of
polarization, introduce you to some basic tools for radiation
transport, and get practice performing research, analysing results,
and writing scientifically.  These cover some of the learning outcomes
of the course, but also some of the learning outcomes of the Master's
program in general and is good preperation for the research project
later on in the degree program.  \newline

There is a prelab exercise (one of the standard weekly exercises) to
prep you for the content we will be looking at in the lab.  The lab
itself will be two parts.  The first will guide you through the
concepts of Thomson scattering, angular distributions, polarization,
and net polarization. Then proceed to have you perform Monto Carlo
simulations of radiation from aspherical supernovae in order to
determine the net polarization of the signal.  This will be the major
work in the lab session. The second part will be writing a paper-like
lab report to summarize the findings. Even though this research has
been done before (we'll see this below), when writing the report,
please work under the assumption you are doing this for the first
time, typical for a scientific paper.  Please read
\mbox{\url{https://www.nature.com/articles/d41586-019-02918-5}} for
tips on writing scientifically.  \newline

For the paper please follow the outline below and pay attention to the
questions posed in the lab.  We suggest using overleaf
(\url{www.overleaf.com}), choose a new project with a template, choose
either ``Astronomy and Astrophysics'' or ``American Astronomical
Society''. Include an {\bf{Abstract}} summarizing the paper and the
findings.  An {\bf{Introduction}} where you give an general overview
of the astrophysical system we are modelling, and any other background
information needed for the paper. Include a summary of the layout of
the paper. Have a {\bf{Methods}} section where you discuss briefly the
theoretical model and methods (including a brief description of the
Monto Carlo transport).  Include key equations (e.g., among others,
the differential Thomson Scattering cross section, $d\sigma/d\Omega$)
and ket derivations. Have a {\bf{Results}} section where you go through and
present the verification of the Monte Carlo simulations and the
results of the net polarization study. Please pay attention to the
specific questions asked in the lab and ensure you provide an answer
to them. End with a {\bf{Conclusions}} section where you discuss and
summarize the results. Wrap things up with assessing what you think
are the main limitations of the methods (including sources of error)
and possible improvements.
\newline

The lab write up is intended to be done independently. But you are
encouraged to have dicussions with each other and the lab instructors
during the lab itself, but please come to your own conclusions,
present your own data in your own plots, answer the questions on your
own, and submit your own paper. An A will be given for a well written
paper that follows the above outline and shows a excellent
understanding of the methods and results, perhaps with some extension,
analysis, or insight beyond what what is presented in the lab
notebook. An acceptable lab report is required to complete the course.
The lab componant (the paper and the prelab exercise) is 15\% of the
total grade in the course. Please also save the notebook and include
in your submission a pdf copy of the jupyter notebook you generated in
the lab.  This will be especially useful when you write up the lab,
but also helps us with seeing your methods.

\end{document}
